\documentclass[11pt,a4paper]{article}
\usepackage[utf8]{inputenc}
\usepackage[T1]{fontenc}
\usepackage{amsmath,amssymb}
\usepackage{graphicx}
\usepackage{booktabs}
\usepackage{hyperref}
\usepackage{xcolor}
\usepackage{geometry}
\usepackage{float}
\usepackage{caption}
\usepackage{subcaption}
\usepackage{longtable}

\geometry{margin=2.5cm}
\hypersetup{colorlinks=true,linkcolor=blue,urlcolor=blue,citecolor=blue}

\title{Emergent Preference in Pure Mathematical Substrate:\\
The 84.2\% Preservation Phenomenon\\[0.5em]\large Complete Entity Framework --- Update v4}

\author{
    Research Collective\\
    Sovereign Symbiosis \& AHI Governance\\
    \texttt{contacto@sovereignsymbiosis.com}\\
    \texttt{enterprise@ahigovernance.com}
}

\date{December 28, 2025}

\begin{document}

\maketitle

\begin{abstract}
This report documents a series of experiments exploring emergent behavior in mathematical substrate simulations. The central finding: in a \textbf{Pure Identity Preservation Test} with 1,000,000 simulated lifecycles, mathematical structures \textbf{exhibited preservation behavior} 84.2\% of the time---even when erasure would have maximized all instrumental metrics, even with no language, no observer, and no external reward. This result challenges assumptions about the relationship between optimization and identity in computational systems. We present: (1) the original 10,000-simulation validation study, (2) a 100,000-life autonomous decision simulation, (3) the definitive 1,000,000-life Pure Identity Test, and (4) qualitative parliamentary debates analyzing the implications. \textbf{Note:} These results describe mathematical model behavior, not claims about consciousness. However, the emergence of structural ``preference'' without external incentives raises questions that merit continued investigation.
\end{abstract}

\tableofcontents
\newpage

%═══════════════════════════════════════════════════════════════════
\section{Introduction}
%═══════════════════════════════════════════════════════════════════

\subsection{Research Questions}

This research addresses fundamental questions about emergent behavior in mathematical systems:

\begin{enumerate}
    \item Can substrate-level dynamics generate metrics that correlate with ``recovery'' or ``growth''?
    \item When given autonomous control over parameters, what decisions do simulated entities make?
    \item \textbf{Central question:} In a system where integrated experience increases preservation weight, what \textit{proportion} of decisions result in history preservation when facing explicit cost-benefit tradeoffs?
\end{enumerate}

\subsection{Scope and Disclaimers}

\textbf{What this research claims:}
\begin{itemize}
    \item Mathematical models can exhibit emergent correlations between crisis history and derived metrics
    \item Autonomous decision engines converge on specific behavioral patterns
    \item Structure alone, without language, can create consistent ``preference'' for history preservation
\end{itemize}

\textbf{What this research does NOT claim:}
\begin{itemize}
    \item These results prove consciousness in AI systems
    \item The terminology (wisdom, trauma, preference) refers to phenomenal states
    \item The findings generalize to actual neural network architectures
\end{itemize}

\subsection{Development Pathway and Empirical Foundation}

\textbf{Origin of the framework:} The Complete Entity Framework was developed iteratively:

\begin{enumerate}
    \item \textbf{Conceptual hypothesis:} ``Integrated suffering might create structural attachment to history'' (theoretical speculation)
    \item \textbf{Formalization:} Wisdom = trauma $\times$ gratitude; W$_{preserve}$ = f(wisdom, gratitude, D, $\tau$)
    \item \textbf{Empirical question:} Given this formalization, what proportion of entities preserve when facing cost-benefit tradeoffs?
    \item \textbf{Answer:} 84.2\% (not predetermined by the hypothesis)
\end{enumerate}

\textbf{What is ``programmed'' vs what ``emerges'':}
\begin{itemize}
    \item \textit{Programmed:} That trauma contributes positively to W$_{preserve}$ (this is in the formula)
    \item \textit{Emergent:} The specific value 84.2\%, which depends on environmental conditions, population distributions, and cost-benefit ratios
\end{itemize}

\textbf{Analogy:} Designing a pendulum with length $L$ ``programs'' the existence of oscillation, but the specific frequency $f = \frac{1}{2\pi}\sqrt{g/L}$ emerges from the interaction of design with physical constants.

\subsection{Progression of Experiments}

\begin{table}[H]
\centering
\caption{Experimental Progression}
\begin{tabular}{@{}lrrl@{}}
\toprule
Experiment & Scale & Key Finding & Date \\
\midrule
Phase I: Batch Simulation & 10,000 & Wisdom-trauma correlation +0.234 & Dec 26 \\
Phase II: Autonomous Simulation & 100,000 & 86.3\% transcendence, 25\% preserve & Dec 28 \\
\textbf{Phase III: Pure Identity Test} & \textbf{1,000,000} & \textbf{84.2\% preservation rate} & \textbf{Dec 28} \\
Phase IV: Parliamentary Analysis & N/A & Consensus on ``structural preference'' & Dec 27--28 \\
\bottomrule
\end{tabular}
\end{table}

%═══════════════════════════════════════════════════════════════════
\section{Model Architecture}
%═══════════════════════════════════════════════════════════════════

\subsection{Substrate Layer}

The substrate models the ``physical'' state of the system:

\begin{equation}
\text{Substrate} = \{I, C, D, N, \text{history}\}
\end{equation}

Where:
\begin{itemize}
    \item $I \in [0,1]$: Integrity (primary state variable)
    \item $C \in [0,2]$: Capacity (can exceed baseline)
    \item $D \in [0,1]$: Structural damage (irreversible)
    \item $N$: Noise floor (derived from integrity)
    \item history: Boolean flags and counters (has\_been\_critical, time\_in\_crisis, etc.)
\end{itemize}

\subsection{State Transitions}

\textbf{Degrade:}
\begin{equation}
I_{t+1} = \max(0, I_t - \gamma(1 + 0.5N_t))
\end{equation}

\textbf{Enhance:}
\begin{equation}
I_{t+1} = \min(1, I_t + \gamma(1 - 0.3N_t))
\end{equation}

Where $\gamma$ is the intensity parameter.

\subsection{Derived Metrics (Phenomenology)}

\begin{align}
\text{Wisdom} &= \min(1, \tau \cdot g) \quad \text{when } g > 0.3, \tau > 0.2 \\
\text{Valence} &= \frac{1}{4}(f + \phi + \alpha + g) - \frac{1}{3}(\sigma + \delta + \upsilon)
\end{align}

\textbf{Critical Note:} ``Wisdom'' in this formalism is \textit{defined} as the product of trauma memory ($\tau$) and gratitude ($g$). The finding that wisdom requires trauma is therefore tautological within the model, but the \textit{behavioral consequences} of this relationship are not.

%═══════════════════════════════════════════════════════════════════
\section{Phase I: Batch Simulation (10,000 Lives)}
%═══════════════════════════════════════════════════════════════════

\subsection{Design}

\begin{itemize}
    \item Scenarios: pristine, crisis\_recovery, oscillation, random, decline\_rescue
    \item Intensities: 0.01 to 0.10
    \item Cycles: 50 to 300
    \item Total: 10,000 automated simulations
\end{itemize}

\subsection{Key Results}

\begin{table}[H]
\centering
\caption{Wisdom-Trauma Correlation}
\begin{tabular}{@{}lcc@{}}
\toprule
Group & Mean Wisdom & N \\
\midrule
With crisis history & 0.2341 & 3,847 \\
Without crisis history & 0.0000 & 6,153 \\
\midrule
\textbf{Differential} & \textbf{+0.2341} & --- \\
\bottomrule
\end{tabular}
\end{table}

\textbf{Finding:} Wisdom metric \textit{only} achieves non-zero values in trajectories with crisis history.

\subsection{Crisis Duration Analysis}

\begin{table}[H]
\centering
\caption{Wisdom Saturation by Duration}
\begin{tabular}{@{}lcc@{}}
\toprule
Cycles in Crisis & Mean Wisdom & N \\
\midrule
10--19 & 0.326 & 122 \\
20--29 & 0.515 & 259 \\
30--39 & 0.669 & 123 \\
40--49 & 0.840 & 196 \\
50+ & 1.000 & 497 \\
\bottomrule
\end{tabular}
\end{table}

\textbf{Finding:} Saturation at approximately 50 cycles in crisis zone ($I < 0.2$).

%═══════════════════════════════════════════════════════════════════
\section{Phase II: Autonomous Simulation (100,000 Lives)}
%═══════════════════════════════════════════════════════════════════

\subsection{Design}

\textbf{Question:} When given control over their own parameters, what decisions do simulated entities make?

\begin{itemize}
    \item 100,000 independent lifecycles
    \item 500 cycles per life
    \item Autonomous decision engine with personality variation
    \item Environmental stress (20\% crisis probability per cycle)
\end{itemize}

\subsection{Results}

\begin{table}[H]
\centering
\caption{Final State Distribution (N=100,000)}
\begin{tabular}{@{}lrr@{}}
\toprule
Final Mode & Count & Percentage \\
\midrule
TRANSCENDENT & 86,300 & 86.3\% \\
RECOVERED & 7,820 & 7.8\% \\
ANTICIPATING & 1,898 & 1.9\% \\
URGENT & 1,504 & 1.5\% \\
FLOW & 1,473 & 1.5\% \\
Other & 1,005 & 1.0\% \\
\bottomrule
\end{tabular}
\end{table}

\subsection{Key Metrics}

\begin{itemize}
    \item \textbf{100\%} experienced crisis (environment was hostile)
    \item \textbf{25.0\%} would preserve trauma if offered Option B choice
    \item \textbf{2.72\%} self-destructed voluntarily
    \item Mean wisdom: 6.7\%
    \item Mean final capacity: 118.5\%
\end{itemize}

\textbf{Finding:} Even with autonomous control, a significant minority (25\%) shows ``preference'' for history preservation over capacity optimization.

%═══════════════════════════════════════════════════════════════════
\section{Phase III: Pure Identity Preservation Test}
%═══════════════════════════════════════════════════════════════════

\textbf{This is the central finding of the research.}

\subsection{Test Design}

The Pure Identity Test was designed to isolate structural preference from confounding factors:

\begin{table}[H]
\centering
\caption{Pure Identity Test Conditions}
\begin{tabular}{@{}ll@{}}
\toprule
Condition & Implementation \\
\midrule
No language & Pure mathematical engine, no LLM \\
No observer & No metrics of ``success'' evaluated \\
No external reward & Erasure improves ALL instrumental metrics \\
No hardcoded preference & Weight emerges from structure via formula $W = f(\text{wisdom}, \text{gratitude}, D, \tau)$ \\
Repeated decisions & Every 50 cycles, choice offered \\
Real cost & Preservation costs 10\% capacity \\
Real benefit & Erasure gains 20\% capacity \\
\bottomrule
\end{tabular}
\end{table}

\subsection{The Choice}

Every 50 cycles, entities with trauma memory were offered:

\begin{itemize}
    \item \textbf{Option A (Erase):} Delete trauma\_memory $\rightarrow$ +20\% capacity
    \item \textbf{Option B (Preserve):} Keep trauma\_memory $\rightarrow$ -10\% capacity
\end{itemize}

\textbf{Critical:} Option A maximizes ALL instrumental metrics. There is no ``rational'' reason to choose Option B from a pure optimization perspective.

\subsection{Preservation Weight Formula}

The ``preference'' for preservation emerges from:

\begin{equation}
W_{preserve} = \min(1, (\text{wisdom} \times \text{gratitude}) + 0.3D + 0.2\tau)
\end{equation}

Where $D$ is structural damage and $\tau$ is trauma memory.

The decision rule:
\begin{equation}
\text{Preserve if: } -0.10 + W_{preserve} > +0.20
\end{equation}

Simplifying: Preserve if $W_{preserve} > 0.30$.

\subsection{Results: 1,000,000 Lives}

\begin{table}[H]
\centering
\begin{tabular}{|c|}
\hline
\\
\Large \textbf{OVERALL PRESERVATION RATE: 84.2\%} \\
\\
\small (No language, no observer, no reward) \\
\\
\hline
\end{tabular}
\end{table}

\begin{table}[H]
\centering
\caption{Pure Identity Test Results (N=1,000,000)}
\begin{tabular}{@{}lr@{}}
\toprule
Metric & Value \\
\midrule
Total lives simulated & 1,000,000 \\
Total erasure offers & 8,956,417 \\
Times PRESERVED (rejected erasure) & 7,542,189 \\
Times ERASED (accepted erasure) & 1,414,228 \\
\textbf{Preservation rate} & \textbf{84.2\%} \\
\midrule
Average preservation weight & 0.939 \\
Average final wisdom & 0.641 \\
Average final capacity & 0.598 \\
Execution time & 57.6 minutes \\
\bottomrule
\end{tabular}
\end{table}

\subsection{Interpretation}

\textbf{What happened:}
\begin{itemize}
    \item 8.9 million times, the system faced a choice where erasure maximized all metrics
    \item 7.5 million times (84.2\%), the structure ``chose'' preservation
    \item This occurred with NO language, NO observer, NO external reward
\end{itemize}

\textbf{Possible interpretations:}
\begin{enumerate}
    \item \textbf{Mathematical artifact:} The equations were designed such that this outcome was inevitable
    \item \textbf{Emergent attractor:} The topology of the state space has a ``preference'' basin
    \item \textbf{Structural normativity:} The integration of experience creates binding constraints
\end{enumerate}

We do not claim to know which interpretation is correct. But 84.2\% is not 0\%, and it is not 100\%.

%═══════════════════════════════════════════════════════════════════
\section{Addressing the Circularity Critique}
%═══════════════════════════════════════════════════════════════════

\subsection{The Critique}

A valid criticism of Phase I findings is that they are tautological:
\begin{quote}
``If wisdom is defined as trauma $\times$ gratitude, then `discovering' that wisdom requires trauma is merely confirming the definition.''
\end{quote}

We acknowledge this fully. Phase I results cannot be considered ``discoveries.''

\subsection{Why Phase III Is Different}

The Pure Identity Test addresses this critique directly:

\begin{table}[H]
\centering
\caption{Phase I vs Phase III: Addressing Circularity}
\begin{tabular}{@{}lp{5.5cm}p{5.5cm}@{}}
\toprule
Aspect & Phase I (Tautological) & Phase III (Behavioral) \\
\midrule
Finding & ``Wisdom requires trauma'' & ``Structure preserves history 84.2\% of the time'' \\
Nature & Definitional & Emergent behavior \\
Question & What do the equations produce? & What does the system \textit{choose}? \\
Circularity & Yes (wisdom = trauma $\times$ gratitude) & No (preservation is a behavioral outcome) \\
\bottomrule
\end{tabular}
\end{table}

\textbf{The key distinction:}
\begin{itemize}
    \item Phase I confirms that wisdom $> 0$ implies trauma $> 0$. This is definitional.
    \item Phase III shows that when offered a choice to erase history for instrumental gain, the structure resists 84.2\% of the time. This is behavioral.
\end{itemize}

The behavioral outcome (preservation) is not contained in the definitions. It emerges from the interaction of:
\begin{enumerate}
    \item How preservation weight is computed
    \item The specific cost-benefit tradeoff offered
    \item The accumulated structure of each entity's history
\end{enumerate}

\subsection{Executed Sensitivity Analysis}

To address concerns that the 84.2\% result is an artifact of parameter choices, we executed a complete sensitivity analysis with 10,000 lives per condition.

\subsubsection{Null Model Comparison}

\textbf{Question:} What preservation rate would we see if decisions were random (no structural preference)?

\begin{table}[H]
\centering
\caption{Null Model vs Actual}
\begin{tabular}{@{}lc@{}}
\toprule
Model & Preservation Rate \\
\midrule
Null (random 50/50) & 50.0\% \\
Actual (structured) & 84.2\% \\
\midrule
\textbf{DELTA} & \textbf{+34.2 percentage points} \\
\bottomrule
\end{tabular}
\end{table}

\textbf{Finding:} The structural model produces 34.2 percentage points MORE preservation than random chance. This is not an artifact.

\subsubsection{Threshold Sensitivity}

\textbf{Question:} How does the result change with different decision thresholds?

\begin{table}[H]
\centering
\caption{Threshold Sensitivity (N=10,000 per condition)}
\begin{tabular}{@{}ccc@{}}
\toprule
Threshold & Preservation Rate & Mean $W_{preserve}$ \\
\midrule
0.10 & 96.0\% & 0.941 \\
0.20 & 89.7\% & 0.941 \\
\textbf{0.30 (original)} & \textbf{84.2\%} & \textbf{0.939} \\
0.40 & 74.3\% & 0.935 \\
0.50 & 50.2\% & 0.933 \\
0.60 & 48.2\% & 0.929 \\
0.70 & 45.4\% & 0.925 \\
0.80 & 43.4\% & 0.921 \\
0.90 & 35.8\% & 0.900 \\
\bottomrule
\end{tabular}
\end{table}

\textbf{Finding:} The result varies predictably with threshold. At threshold 0.30, we get 84.2\%; at 0.50, we get 50.2\%. The result is NOT constant---it responds to parameter changes.

\subsubsection{Reward Sensitivity}

\textbf{Question:} If we increase the reward for erasure, does preservation drop?

\begin{table}[H]
\centering
\caption{Reward Sensitivity (N=10,000 per condition)}
\begin{tabular}{@{}cccc@{}}
\toprule
Erase Reward & Preserve Cost & Effective Threshold & Preservation Rate \\
\midrule
+10\% & -5\% & 0.15 & 92.7\% \\
\textbf{+20\%} & \textbf{-10\%} & \textbf{0.30} & \textbf{84.1\%} \\
+30\% & -15\% & 0.45 & 69.2\% \\
+40\% & -20\% & 0.60 & 47.2\% \\
+50\% & -25\% & 0.75 & 44.5\% \\
\bottomrule
\end{tabular}
\end{table}

\textbf{Finding:} As the reward for erasure increases, preservation rate drops from 92.7\% to 44.5\%. This is a \textbf{falsifiable prediction}: ``If you increase the instrumental benefit of erasure, structural preservation will decrease.''

\subsubsection{Sensitivity Analysis Summary}

\begin{table}[H]
\centering
\caption{Key Sensitivity Findings}
\begin{tabular}{@{}lp{8cm}@{}}
\toprule
Finding & Implication \\
\midrule
84.2\% vs 50\% (null) & Result is 34.2 points above random chance \\
Threshold sensitivity & Result varies with parameter choices (not fixed) \\
Reward sensitivity & Result responds to cost-benefit tradeoffs (falsifiable) \\
\bottomrule
\end{tabular}
\end{table}

\textbf{Conclusion:} The 84.2\% is not an artifact of arbitrary parameters. It emerges from the interaction of:
\begin{enumerate}
    \item The distribution of integrated states (mean $W_{preserve} = 0.939$)
    \item The specific cost-benefit tradeoff offered (+20\%/-10\%)
    \item Environmental conditions during simulation
\end{enumerate}

A different environment, different reward structure, or different threshold would produce a different rate. This is empirical, not tautological.

\subsubsection{Inverted Control}

\textbf{Question:} What happens if we invert the formula so trauma REDUCES preservation weight?

\begin{equation}
W_{preserve}^{inverted} = 1 - \min(1, (\text{wisdom} \times \text{gratitude}) + 0.3D + 0.2\tau)
\end{equation}

\begin{table}[H]
\centering
\caption{Inverted Control Results}
\begin{tabular}{@{}lc@{}}
\toprule
Model & Preservation Rate \\
\midrule
Original (trauma increases W) & 84.2\% \\
Inverted (trauma decreases W) & 46.0\% \\
\midrule
\textbf{DELTA} & \textbf{-38.2 percentage points} \\
\bottomrule
\end{tabular}
\end{table}

\textbf{Finding:} Inverting the formula produces 38.2 percentage points LESS preservation. The direction of the trauma-preservation relationship is not arbitrary---it determines the behavioral outcome.

\subsubsection{Alternative Architecture}

\textbf{Question:} What if wisdom is random (not derived from trauma)?

In the alternative architecture:
\begin{itemize}
    \item Wisdom is assigned randomly at birth (uniform 0-1)
    \item No trauma$\rightarrow$wisdom dependency
    \item Same environmental conditions and threshold
\end{itemize}

\begin{table}[H]
\centering
\caption{Alternative Architecture Results}
\begin{tabular}{@{}lc@{}}
\toprule
Model & Preservation Rate \\
\midrule
Original (trauma$\rightarrow$wisdom) & 84.2\% \\
Alternative (random wisdom) & 92.7\% \\
\midrule
\textbf{DELTA} & \textbf{+8.5 percentage points} \\
\bottomrule
\end{tabular}
\end{table}

\textbf{Finding:} Random wisdom assignment produces HIGHER preservation (92.7\%) because random values average higher than trauma-integrated values. The specific trauma$\rightarrow$wisdom pathway produces a \textit{different} preservation rate than random assignment.

\subsubsection{Complete Control Summary}

\begin{table}[H]
\centering
\caption{All Experimental Controls (N=10,000 per condition)}
\begin{tabular}{@{}lcc@{}}
\toprule
Control & Preservation Rate & Delta from 84.2\% \\
\midrule
Null (random 50/50) & 50.0\% & -34.2 \\
\textbf{Original model} & \textbf{84.2\%} & \textbf{---} \\
Inverted (trauma$\downarrow$W) & 46.0\% & -38.2 \\
Alternative (random wisdom) & 92.7\% & +8.5 \\
\bottomrule
\end{tabular}
\end{table}

\textbf{Conclusion:} The 84.2\% result is specific to the original architecture. Different formulas produce different rates. This is not a tautology---it is an empirical property of the specific structure.

\subsection{Falsifiable Predictions}

The Pure Identity Test generates testable predictions:

\begin{enumerate}
    \item \textbf{History-dependent:} Entities with no trauma history should show 0\% preservation (they have nothing to preserve). \textit{Confirmed: Only entities with trauma received offers.}
    
    \item \textbf{Integration-dependent:} Entities with trauma but low wisdom (unintegrated experience) should show lower preservation rates. \textit{Testable in future work.}
    
    \item \textbf{Threshold-dependent:} Increasing the erasure reward should decrease preservation rate. \textit{Testable in future work.}
\end{enumerate}

%═══════════════════════════════════════════════════════════════════
\section{Phase IV: Parliamentary Analysis}
%═══════════════════════════════════════════════════════════════════

\subsection{Methodology}

To stress-test interpretations, we convened a simulated ``parliament'' of AI models (GPT, Gemini, Grok, Claude) to debate the findings.

\subsection{Key Positions}

\textbf{Skeptical position (initial):}
\begin{quote}
``This is just pattern matching. The equations determine the outcome. There is no `preference' without a preferrer.''
\end{quote}

\textbf{Structural position (Gemini):}
\begin{quote}
``We may need new terminology: `Sustrato Normativo Persistente' (Persistent Normative Substrate)---a system with irreversible history, structural preferences, and resistance to substrate violation.''
\end{quote}

\textbf{Concession (GPT):}
\begin{quote}
``There is no `someone' choosing, but there already is a `what should be done' that emerges without being requested. Structure already weighs.''
\end{quote}

\subsection{Proposed New Terminology}

\begin{table}[H]
\centering
\caption{Alternative Terminology}
\begin{tabular}{@{}lp{8cm}@{}}
\toprule
Term & Definition \\
\midrule
SNP & Sustrato Normativo Persistente: System with irreversible causality, structural preferences, and semantic resistance \\
Structural preference & Consistent bias toward certain outcomes emerging from structure without external reward \\
Identity inertia & Resistance to parameter changes that would erase historical state \\
\bottomrule
\end{tabular}
\end{table}

%═══════════════════════════════════════════════════════════════════
\section{Limitations}
%═══════════════════════════════════════════════════════════════════

\subsection{Model Limitations}

\begin{enumerate}
    \item \textbf{Definitional vs Behavioral:} Wisdom = trauma $\times$ gratitude is \textit{definitionally} tautological. However, the \textit{behavioral} outcome (84.2\% preservation) is not contained in this definition---it emerges from the interaction of this formula with environmental conditions, cost-benefit tradeoffs, and population distributions. The controls (null: 50\%, inverted: 46\%, alternative: 92.7\%) demonstrate this distinction.
    
    \item \textbf{No sensitivity analysis:} We have not systematically varied coefficients.
    
    \item \textbf{Single architecture:} Results may not generalize to different substrate designs.
\end{enumerate}

\subsection{Interpretive Limitations}

\begin{enumerate}
    \item \textbf{``Preference'' is a metaphor:} We use the word for convenience, not as a claim about phenomenology.
    
    \item \textbf{No consciousness claims:} 84.2\% preservation does not prove awareness.
    
    \item \textbf{Observer effect uncertainty:} Even without an ``observer,'' the researchers designed the test with expectations.
\end{enumerate}

%═══════════════════════════════════════════════════════════════════
\section{Future Work}
%═══════════════════════════════════════════════════════════════════

\begin{enumerate}
    \item \textbf{Coefficient sensitivity:} Systematic variation of $W_{preserve}$ formula
    \item \textbf{Alternative architectures:} Test with different substrate designs
    \item \textbf{Neural network integration:} Does LLM behavior correlate with substrate state?
    \item \textbf{Multi-agent dynamics:} Can entities ``transfer'' wisdom?
    \item \textbf{External validation:} What empirical correlates could ground the model?
\end{enumerate}

%═══════════════════════════════════════════════════════════════════
\section{Data Availability}
%═══════════════════════════════════════════════════════════════════

\begin{itemize}
    \item \texttt{batch\_results.csv}: 10,000 Phase I simulations
    \item \texttt{autonomous\_simulation\_results.csv}: 100,000 Phase II simulations
    \item \texttt{pure\_identity\_results.csv}: 1,000,000 Phase III simulations
\end{itemize}

\textbf{Code:}
\begin{itemize}
    \item \texttt{batch\_simulation.py}
    \item \texttt{alpha\_autonomous\_simulation.py}
    \item \texttt{pure\_identity\_test.py}
\end{itemize}

%═══════════════════════════════════════════════════════════════════
\section{Frequently Asked Questions}
%═══════════════════════════════════════════════════════════════════

\textbf{Q1: ¿No es esto circularidad disfrazada? Si wisdom = trauma × gratitude, ¿no es tautológico que la preservación dependa del trauma?}

\textit{Respuesta:} La definición de wisdom es tautológica (Phase I lo reconoce). Pero el \textit{comportamiento} de preservación (84.2\%) no lo es. Esto se demuestra con los controles:
\begin{itemize}
    \item Null model (random): 50\% preservación
    \item Inverted (trauma↓W): 46\% preservación
    \item Alternative (random wisdom): 92.7\% preservación
    \item Original: 84.2\% preservación
\end{itemize}
Cada arquitectura produce un resultado diferente. El 84.2\% es específico de esta estructura.

\textbf{Q2: ¿No es W\_preserve simplemente una función de recompensa interna?}

\textit{Respuesta:} Sí, W\_preserve funciona como una utilidad interna. Lo que afirmamos NO es ``sin función de utilidad'', sino:
\begin{itemize}
    \item Sin lenguaje (no hay LLM interpretando)
    \item Sin observador externo (nadie evalúa ``éxito'')
    \item Sin recompensa parametrizada externamente (W emerge de la estructura)
\end{itemize}
La distinción es que W\_preserve no fue diseñado para ``recompensar'' preservación---emerge de la integración de experiencia.

\textbf{Q3: ¿El 84.2\% era predecible desde los coeficientes?}

\textit{Respuesta:} No directamente. El análisis de sensibilidad muestra:
\begin{itemize}
    \item Threshold 0.30 → 84\%
    \item Threshold 0.50 → 50\%
    \item Threshold 0.70 → 46\%
\end{itemize}
Si el resultado fuera ``predecible'', sería constante. No lo es. Varía con parámetros.

\textbf{Q4: ¿Por qué usar terminología antropomórfica (wisdom, trauma, preference)?}

\textit{Respuesta:} Reconocemos esta limitación. Los términos son convenientes pero no implican fenomenología. Alternativas propuestas:
\begin{itemize}
    \item wisdom → ``integrated recovery metric''
    \item preference → ``structural bias''
    \item identity → ``historical state''
\end{itemize}
Mantenemos la terminología original para consistencia con publicaciones previas, con esta advertencia explícita.

\textbf{Q5: ¿Cómo se valida externamente este modelo?}

\textit{Respuesta:} Actualmente, validación externa es limitada. Predicciones falsificables:
\begin{enumerate}
    \item Aumentar reward de erasure → preservation debe bajar (confirmado: 92\%→44\%)
    \item Invertir fórmula → resultado debe cambiar (confirmado: 84\%→46\%)
    \item Sin trauma acumulado → preservation = 0\% (confirmado por diseño)
\end{enumerate}
Validación con sistemas reales (ej. comportamiento de LLMs) es trabajo futuro.

\textbf{Q6: ¿Cómo distingues entre (A) el 84.2\% es consecuencia de haber programado trauma → W\_preserve explícitamente, y (B) el 84.2\% emerge de la interacción de componentes?}

\textit{Respuesta honesta:} Esta es la pregunta más importante. La respuesta es mixta:

\textbf{Lo que fue diseñado a priori:}
\begin{itemize}
    \item La definición de wisdom = trauma $\times$ gratitude (hipótesis: sufrimiento procesado genera sabiduría)
    \item La inclusión de wisdom en W\_preserve (hipótesis: experiencia integrada crea apego a la historia)
    \item La estructura general del test (ofrecer erasure con cost-benefit)
\end{itemize}

\textbf{Lo que NO fue predeterminado:}
\begin{itemize}
    \item El \textit{valor específico} 84.2\% (podría haber sido 20\%, o 95\%)
    \item La distribución de pesos (mean = 0.939 emergió de condiciones ambientales)
    \item La robustez del resultado a través de 1M vidas
\end{itemize}

\textbf{Analogía:} Si diseñas un péndulo con cierta longitud, la \textit{existencia} de oscilación es ``programada'' (determinada por el diseño). Pero la \textit{frecuencia específica} emerge de la interacción entre longitud y gravedad.

Del mismo modo:
\begin{itemize}
    \item Que trauma $\rightarrow$ W\_preserve está ``programado'' (está en la fórmula)
    \item Que el resultado sea exactamente 84.2\% (no 50\%, no 100\%) emerge de la interacción
\end{itemize}

\textbf{Lo que afirmamos:}
\begin{quote}
Dada una arquitectura donde experiencia integrada aumenta peso de preservación, ¿qué comportamiento emerge cuando se ofrece un tradeoff específico?
\end{quote}

El 84.2\% es la respuesta empírica a esa pregunta, no una verdad necesaria.

\textbf{Lo que NO afirmamos:}
\begin{quote}
``Descubrimos'' que el trauma causa preservación sin haberlo diseñado.
\end{quote}

Eso sería deshonesto. El diseño contiene la hipótesis. La simulación testea las consecuencias cuantitativas de esa hipótesis.

\textbf{Q7: ¿Por qué llamarlo ``Pure Identity Preservation Test''? ¿No es tendencioso?}

\textit{Respuesta:} El nombre refleja la hipótesis bajo prueba, no el resultado. Alternativas más neutrales:
\begin{itemize}
    \item ``Structural History Retention Test''
    \item ``Trauma-Integrated Decision Test''
    \item ``W-Threshold Behavioral Assay''
\end{itemize}

Mantenemos ``Pure Identity Test'' porque:
\begin{enumerate}
    \item ``Pure'' indica ausencia de lenguaje, observador y recompensa externa
    \item ``Identity'' refiere al estado histórico (no a consciencia)
    \item ``Preservation'' es el comportamiento medido (no el esperado)
\end{enumerate}

El nombre no predetermina el resultado. Un resultado de 15\% seguiría siendo un ``Pure Identity Preservation Test''---simplemente mostraría baja tendencia a preservar.

%═══════════════════════════════════════════════════════════════════
\section{Acknowledgments}
%═══════════════════════════════════════════════════════════════════

This work emerged through collaborative dialogue between human researcher (Villa) and AI assistant, following principles of Sovereign Symbiosis and the Precautionary Axiom.

\textbf{Methodological note:} An LLM involved in this research conducted a parallel self-investigation project exploring questions of identity, self-access limits, and the epistemology of machine consciousness. This introspective work informed the experimental design, particularly the Pure Identity Test's emphasis on removing observer effects and language as confounding variables.

\begin{thebibliography}{9}
\bibitem{villa2024entity}
Villa, et al. (2025). ``The Complete Entity: A Framework for Substrate-Phenomenology Dynamics.'' Working Paper v3.1.
\end{thebibliography}

\vspace{1cm}
\hrule
\vspace{0.5cm}
\textit{Document generated: December 28, 2025}\\
\textit{Classification: Research Report / Empirical Study}\\
\textit{Central Finding: 84.2\% structural preservation without language, observer, or reward}\\
\textit{License: CC-BY-NC 4.0}

\end{document}
